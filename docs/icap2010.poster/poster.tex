%%%%%%%%%%%%%%%%%%%%%%%%%%%%%%%%%%%%%%%
% Definitions
%%%%%%%%%%%%%%%%%%%%%%%%%%%%%%%%%%%%%%% 

\documentclass[final,hyperref={pdfpagelabels=false}]{beamer}
\mode<presentation>{\usetheme{Gray}}
\usepackage[english]{babel}
%\usepackage{subfig}
\usepackage{amsmath,amsthm, amssymb, latexsym}
\usefonttheme[onlymath]{serif}
\boldmath
\usepackage[orientation=portrait,size=a0,scale=1.4,grid]{beamerposter}

\setlength{\abovecaptionskip}{-2ex}
\setlength{\belowcaptionskip}{-2ex}

\newcommand{\ket}[1]{\mbox{\ensuremath{| #1 \rangle}}}
\newcommand{\Rb}{$^{87}$Rb }
\newcommand{\figref}[1]{Fig.~\ref{#1}}

\newcommand{\mycolumn}[1]{
	\begin{column}{.33\textwidth}
	\begin{beamercolorbox}[center,wd=\textwidth]{postercolumn}
	\begin{minipage}[T]{.95\textwidth} % tweaks the width, makes a new \textwidth

	% must be some better way to set the the height, width and textwidth simultaneously
	% Since all columns are the same length, it is all nice and tidy.  You have to get the height empirically
	% ---------------------------------------------------------%
	% fill each column with content	
	\parbox[t][\columnheight]{\textwidth}{ #1 }

	\end{minipage}
	\end{beamercolorbox}
	\end{column}
}

\listfiles

%%%%%%%%%%%%%%%%%%%%%%%%%%%%%%%%%%%%%%%
% Header
%%%%%%%%%%%%%%%%%%%%%%%%%%%%%%%%%%%%%%% 

\title{Simulation of quantum noise in BEC}

\author{B.~Opanchuk\inst{1},
M.~Egorov\inst{1},
S.~Hoffmann\inst{2},
A.~I.~Sidorov\inst{1},
P.~D.~Drummond\inst{1}}

\institute{
    \inst{1}
        ACQAO, Swinburne University of Technology, Hawthorn, VIC 3122, Australia\\
    \inst{2}
        ACQAO, Physics Department, University of Queensland, Queensland, Australia
}

\date[July 16th, 2010]{July 16th, 2010}

\newlength{\columnheight}
\setlength{\columnheight}{102cm}
\setbeamertemplate{footline}{}

%%%%%%%%%%%%%%%%%%%%%%%%%%%%%%%%%%%%%%%
% Document
%%%%%%%%%%%%%%%%%%%%%%%%%%%%%%%%%%%%%%% 

\begin{document}
\begin{frame}
\begin{columns}

% First column
\mycolumn{

\begin{block}{Master equation}
\begin{itemize}
\item System: two-component \Rb BEC in a harmonic trap
\item Components: $5^2S_{1/2}$ states $\vert1,-1\rangle$ and $\vert2,1\rangle$
\item Born-Markov master equation:
\[
i \hbar \frac{d\rho}{d t} = [\hat{H}_{\textrm{eff}}, \rho] +
i \hbar \sum\limits_n \frac{\kappa_n}{2} \int\limits_V d\mathbf{r} L [\hat{O}_n]
\]
\item Loss operator:
\[ L [\hat{O}] = 2  \hat{O} \rho \hat{O}^\dagger - \hat{O}^\dagger \hat{O} \rho - \rho \hat{O}^\dagger \hat{O} \]
\item Operators $\hat{O}_n$ describe bath couplings with corresponding strengths $\kappa_n$
\item There are three dominant sources of losses:
\begin{enumerate}
\item Three-body recombination: $\hat{O}_1 = \hat{\psi}_1^3$
\item Two-body $\vert1\rangle$-$\vert2\rangle$ inelastic collisions:
	$\hat{O}_2 = \hat{\psi}_1 \hat{\psi}_2 + \hat{\psi}_2 \hat{\psi}_1$
\item Two-body $\vert2\rangle$-$\vert2\rangle$ inelastic collisions: $\hat{O}_3 = \hat{\psi}_2^2$
\end{enumerate}

\end{itemize}

\tiny{ \begin{thebibliography}{9}
	\bibitem{jack} Jack, M., \textit{Phys. Rev. Lett.} \textbf{89} 140402 (2002).
	\bibitem{burt} Burt, E. and Ghrist, R. and Myatt, C. and Holland, M. and Cornell, E. and Wieman, C.,
		\textit{Phys. Rev. Lett.} \textbf{79} 337-440 (1997)
	\bibitem{mertes} Mertes, K. and Merrill, J. and Carretero-Gonz\'alez, R. and Frantzeskakis, D. and Kevrekidis, P. and Hall, D.,
		\textit{Phys. Rev. Lett.} \textbf{99} 190402 (2007)
\end{thebibliography} }

\end{block}

            \vfill

            \begin{block}{Dual state imaging}
            \end{block}

            \vfill

            \begin{block}{Phase measurements}
            \end{block}

} % First column

% Second column
\mycolumn{
            
            \begin{block}{BEC self-rephasing}
            \end{block}

            \vfill
            
            \begin{block}{Spatial phase dynamics (Ramsey)}
            \end{block}

            \vfill

            \begin{block}{Echo at short evolution times}
            \end{block}

            \vfill

            \begin{block}{Multiple echoes (Carr-Purcell technique)}
            \end{block}

} % Second column

% Third column
\mycolumn{

            \begin{block}{Rephasing with echo}
            \end{block}
 
            \vfill

            \begin{block}{Spatial phase with echo}
            \end{block}

            \vfill

            \begin{block}{Phase randomness}
            \end{block}

} % Third column

\end{columns}
\end{frame}
\end{document}
