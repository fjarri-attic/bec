\documentclass[12pt,notitlepage]{report}
\usepackage{indentfirst}
\usepackage[pdftex]{graphicx}
\usepackage{subfig}

\title{Two-component BEC dynamics calculations using split-step Fourier method}
\author{Bogdan Opanchuk}

\begin{document}

We are looking for steady state of the system with hamiltonian:
\[ 
\hat{H} = -\frac{\hbar^2}{2m} \frac{\partial^2}{\partial \mathbf{r}^2} + \\
V(\mathbf{r}) + \\
g_{11} \vert \psi(\mathbf{r}) \vert^2 
\]
\[g_{11} = \frac{4 \pi \hbar^2 a_{11}}{m}\]
\[ V(\mathbf{r}) = \frac{m}{2} \left( \omega_x^2 x^2 + \omega_y^2 y^2 + \omega_z^2 z^2 \right) \]
where $V(\mathbf{r})$ is the potential energy for parabolic trap, $g_{11}$ is the interaction coefficient
and $a_{11}$ is the scattering length for atoms in non-excited state.

The steady state should satisfy the time-independent Shrodinger equation:
\[ \hat{H} \psi = \mu \psi \]
To get the first approximation of the state function, we consider the kinetic term to be small as compared to other terms and omit it. The conditions for this operation to be valid will be determined later in this section. Thus we get the simple equation:
\[ \left( V(\mathbf{r}) + g_{11} \vert \psi(\mathbf{r}) \vert^2 \right) \psi(\mathbf{r}) = \mu \psi(\mathbf{r}) \]
which leads us to the state function:
\[ \vert \psi(\mathbf{r}) \vert^2 = \frac{1}{g_{11}} \left( \mu - V(\mathbf{r}) \right), V(\mathbf{r}) \leq \mu \]
\[ \psi(\mathbf{r}) \equiv 0, V(\mathbf{r}) > \mu \]
The condition for $V(\mathbf{r})$ defines the shape of the condensate - it is the ellipsoid with the following radii:
\[ r_i = \sqrt{\frac{2 \mu}{m \omega_i^2}}, i = x, y, z \]

Using the normalisation condition for the steady state function gives us the connection between the number of atoms in the condensate and the chemical potential:
\[ \int\limits_{V} \vert \psi(\mathbf{r}) \vert^2 = N \]
\[ 
\mu = \left( \frac{15 N}{8 \pi} \right)^\frac{2}{5} \\
\left( \frac{m \bar{\omega}^2}{2} \right)^\frac{3}{5} \\
{g_{11}}^\frac{2}{5}
\]
where $\bar{\omega} = \sqrt[3]{\omega_x \omega_y \omega_z}$.

Now we can estimate the conditions necessary to drop the kinetic term from equation. We will do it for one-dimensional case for the sake of simplicity; three-dimensional results will differ only in coefficients of the order of unity:
\[ 
\frac{\hbar^2}{2 m} \vert \frac{d^2 \psi}{d x^2} \vert \ll \\
\left( V + g_{11} \vert \psi \vert^2 \right) \vert \psi \vert
\]
Substituting the expression for $\psi$ and simplifying the inequation:
\[
\frac{\hbar^2}{2 m} \left( \mu + \frac{V}{2} \right) \ll \\
x^2 \left( \mu - V \right)^2 + \frac{2}{m \omega^2} \left( \mu - V \right)^3
\]

Simple (but lengthy) calculation of the first order derivative shows that we need to investigate only cases of $x=0$ and
$x=r_x$. For $x=0$ the condition straightforwardly turns into $\mu \gg \frac{\hbar \omega}{2}$, or just
$\mu \gg \hbar \omega$, since the coefficient of the order of unity is insignificant.
On the other hand, for $x=r_x$ the inequation turns into $\mu \ll 0$ which is apparently always false. This means that
near the edges of the condensate Thomas-Fermi approximation will fail. Fortunately, the density of the particles
there is low, so we can estimate the width $h$ of the "belt" where our first approximation of the state function is
sufficiently incorrect. After replacing $x$ by $r_x - h$ and "$\ll$" by "$\approx$" in inequation, we will get (providing that 
$h \ll r_x$) $h \approx \sqrt{\frac{3 \hbar^2}{m \mu}}$. So, in order the belt to be narrow --- or, in other words
$h \ll r_x$ --- the chemical potential should satisfy the similarly looking condition as for the case of $x=0$: 
$\mu \gg \sqrt{\frac{3}{2}} \hbar \omega$ or just $\mu \gg \hbar \omega$. For three-dimensional case we can safely substitute $\bar{\omega}$ for $\omega$ in the condition.

Let us use some real-life experimental parameters and check how well Thomas-Fermi approximation works.
For three-dimensional trap with frequencies $f_x = f_y = 97.6 \textrm{ Hz}$ and $f_x = 11.96 \textrm{ Hz}$ and
$10^5$ rubidium atoms (which have scattering length $a_{11} = 100.4 a_0$, where $a_0$ is Bohr radius):
\[ \frac{\mu}{\hbar \bar{\omega}} \approx 15.3 \]
This is a pretty good result, which means that Thomas-Fermi approximation produces state close to the real one.
But for lower amount of atoms, say $10^4$, we get:
\[ \frac{\mu}{\hbar \bar{\omega}} \approx 6 \]
which is lower than the order of ten. Therefore, for low amounts of atoms, Thomas-Fermi approximation should be
used with care. 

\end{document}