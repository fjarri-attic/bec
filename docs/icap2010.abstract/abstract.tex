%
% LaTeX template for ICAP 2010 abstracts.
%
\documentclass[10pt]{article}

\usepackage{graphicx}
\usepackage{icap2010}

\begin{document}
%%%%%%%%%   DO NOT MODIFY ANYTHING ABOVE THIS LINE %%%%%%%%

\title{Simulation of quantum noise in BEC}

\begin{authors}
 \author{B.}{Opanchuk}{1},
  \author{M.}{Egorov}{1},
  \author{S.}{Hoffmann}{2},
  \author{A.}{Sidorov}{1},
    \author{P. D.}{Drummond}{1}
\end{authors}

\address{1}{ACQAO Centre, Swinburne University of Technology, Hawthorn, Victoria, Australia}
\address{2}{ACQAO Centre, Physics Department, University of Queensland, Queensland, Australia}


\begintext

Atom interferometry is an important quantum technology, at the heart of many suggested future applications of ultra-cold atomic physics. Bose-Einstein condensates or atom lasers have potential advantages as detectors or sensors, provided one can extract atomic phase information. However, there is a challenge to be met. Unlike photons, atoms can interact rather strongly, causing dephasing. An intimate understanding of quantum many-body dynamics is essential in understanding the precise nature of interaction-induced dephasing in the measurement process. Progress in this technology therefore requires a quantitative theory of atom interferometry.

Coupled Gross-Pitaevskii equations are widely used to simulate the dynamics of BEC. This method is easy and relatively fast, and it allows one to obtain rather accurate predictions for the evolution of cloud structure$^1$. The problem is that this method  produces incorrect results for low number of particles or long evolution times, since it ignores quantum noise. There are ways to improve GPE-based simulations, one of which is the inclusion of quantum noise terms via the Wigner representation. 

Here, we give a simple, yet quantitatively accurate theoretical approach to calculations of atom interferometry using a truncated Wigner representation. This method extends the conventional Gross-Pitaevskii equations describing a Bose condensate to include quantum noise effects. We show through comparison with experimental interferometric measurements on Bose condensates, that the theory predicts observed dephasing results, within experimental errors. No fitting parameters or added technical noise is required in these comparisons.  Importantly, we can clearly demonstrate where phase decay is driven by intrinsic quantum fluctuations due to the effects of a beamsplitter, and where it is driven by trap inhomogeneity or other effects.

The noise can be taken into account by writing the master equation for the condensate and transforming it into one of the quasiprobability representations, like \mbox{positive-P}$^{2, 3}$ or Wigner$^2$. The resulting equation can then be expressed as a set of stochastic equations, which can be solved numerically. We present the results of such simulations for several types of experiments on $^{87}$Rb condensates (namely, two-component clouds of $\vert1,-1\rangle$ and $\vert2,-1\rangle$ states in optical or magnetic traps). In addition, we use the simulation data to revise the values of scattering lengths and loss terms for these components.

\footnotetext[1]{R.~P.~Anderson, C.~Ticknor, A.~I.~Sidorov, B.~V.~Hall,
Phys. Rev. A, {\bf 80}:023603 (2009)}

\footnotetext[2]{M.~J.~Steel, M.~K.~Olsen, L.~I.~Plimak, P.~D.~Drummond, S.~M.~Tan, M.~J.~Collett, D.~F.~Walls and R.~Graham,
Phys. Rev. A, {\bf 58}(6):48244835 (1998)}

\footnotetext[3]{J.~J.~Hope,
Phys. Rev. A, {\bf 64}(5):053608 (2001)}
    
%%%%%%%%%%   DO NOT MODIFY ANYTHING BELOW THIS LINE %%%%%%%%
\end{document}
