%%%%%%%%%%%%%%%%%%%%%%%%%%%%%%%%%%%%%%%
% Definitions
%%%%%%%%%%%%%%%%%%%%%%%%%%%%%%%%%%%%%%% 

\documentclass[final,hyperref={pdfpagelabels=false}]{beamer}
\mode<presentation>{\usetheme{Gray}}
\usepackage[english]{babel}
\usepackage{amsmath,amsthm, amssymb, latexsym}
\usefonttheme[onlymath]{serif}
\boldmath
\usepackage[orientation=portrait,size=a0,scale=1.4,grid]{beamerposter}

\setlength{\abovecaptionskip}{-2ex}
\setlength{\belowcaptionskip}{-2ex}
\setbeamertemplate{bibliography item}{\insertbiblabel}

\newcommand{\ket}[1]{\mbox{\ensuremath{| #1 \rangle}}}
\newcommand{\Rb}{$^{87}$Rb }
\newcommand{\figref}[1]{Fig.~\ref{#1}}

\newcommand{\mycolumn}[1]{
	\begin{column}{.5\textwidth}
	\begin{beamercolorbox}[center,wd=\textwidth]{postercolumn}
	\begin{minipage}[T]{.95\textwidth} % tweaks the width, makes a new \textwidth

	% must be some better way to set the the height, width and textwidth simultaneously
	% Since all columns are the same length, it is all nice and tidy.  You have to get the height empirically
	% ---------------------------------------------------------%
	% fill each column with content	
	\parbox[t][\columnheight]{\textwidth}{ #1 }

	\end{minipage}
	\end{beamercolorbox}
	\end{column}
}

\listfiles

%%%%%%%%%%%%%%%%%%%%%%%%%%%%%%%%%%%%%%%
% Header
%%%%%%%%%%%%%%%%%%%%%%%%%%%%%%%%%%%%%%% 

\title{Simulation of quantum noise in BEC}

\author{B.~Opanchuk\inst{1},
M.~Egorov\inst{1},
S.~Hoffmann\inst{2},
A.~I.~Sidorov\inst{1},
P.~D.~Drummond\inst{1}}

\institute{
    \inst{1}
        ACQAO, Swinburne University of Technology, Hawthorn, VIC 3122, Australia\\
    \inst{2}
        ACQAO, Physics Department, University of Queensland, Queensland, Australia
}

\date[July 16th, 2010]{July 16th, 2010}

\newlength{\columnheight}
\setlength{\columnheight}{102cm}
\setbeamertemplate{footline}{}

%%%%%%%%%%%%%%%%%%%%%%%%%%%%%%%%%%%%%%%
% Document
%%%%%%%%%%%%%%%%%%%%%%%%%%%%%%%%%%%%%%% 

\begin{document}
\begin{frame}
\begin{columns}

% First column
\mycolumn{

\begin{block}{Trapped two-component BEC}

\begin{itemize}
	\item System: two-component \Rb BEC in a harmonic trap
	\item Components: $5^2S_{1/2}$ states $\vert1,-1\rangle$ and $\vert2,1\rangle$
	\item Second-quantized hamiltonian:
\begin{align*}
\begin{split}
\hat{H} = {} & \int d\mathbf{r} \left\{
	\sum\limits^2_{i=1} \hat{\psi}_i^\dagger(\mathbf{r}) \hat{K} \hat{\psi}_i(\mathbf{r}) +
	V_{hf} \hat{\psi}_2^\dagger(\mathbf{r}) \hat{\psi}_2(\mathbf{r}) \right. \\
& \left. + \frac{1}{2} \sum\limits^2_{i=1} \sum\limits^2_{j=1} \int d\mathbf{r}^\prime \,
	\hat{\psi}_i^\dagger (\mathbf{r}) \hat{\psi}_j^\dagger (\mathbf{r}^\prime)
	U_{ij}(\mathbf{r} - \mathbf{r}^\prime)
	\hat{\psi}_i(\mathbf{r}^\prime) \hat{\psi}_j(\mathbf{r}) \right. \\
& \left. + \frac{\hbar}{2} \tilde{\Omega} e^{i \omega t} \left(
		\hat{\psi}_1^\dagger(\mathbf{r}) \hat{\psi}_2(\mathbf{r}) +
		\hat{\psi}_2^\dagger(\mathbf{r}) \hat{\psi}_1(\mathbf{r})
	\right) \right. \\
& \left. + \frac{\hbar}{2} \tilde{\Omega}^* e^{-i \omega t} \left(
		\hat{\psi}_1(\mathbf{r}) \hat{\psi}_2(\mathbf{r})^\dagger +
		\hat{\psi}_2(\mathbf{r}) \hat{\psi}_1(\mathbf{r})^\dagger
	\right)
\right\}
\end{split}
\end{align*}
% FIXME:
% Is the scattering term actually $\hat{\psi}_i^\dagger (\mathbf{r}) \hat{\psi}_j^\dagger (\mathbf{r}^\prime)
%	U_{2b}(\mathbf{r} - \mathbf{r}^\prime)
%	\hat{\psi}_j(\mathbf{r}^\prime) \hat{\psi}_i(\mathbf{r})$?
	\item Noninteracting term $\hat{K} = - \frac{\hbar^2 \nabla^2}{2 m} + V(\mathbf{r})$
	\item Hyperfine splitting energy $V_{\textrm{hf}} / h = 6.8 \textrm{GHz}$
	\item Two-photon coupling constant $\tilde{\Omega} = \Omega e^{i \alpha}$,
		where $\Omega$ is the Rabi frequency
	\item Field annihilation operators $\hat{\psi}_i (\mathbf{r})$ obey the common commutation relations for identical bosons
\end{itemize}

\end{block}

\begin{block}{Restricted basis field}

\begin{itemize}
	\item Basis $\phi_{n}$ of plane waves: $\phi_{n}(\mathbf{r}) = e^{i \mathbf{k}_n \mathbf{r}} / \sqrt{V}$
	\item Decomposed field operators: $\hat{\psi}_i(\mathbf{r}) = \sum\limits_n \phi_{n}(\mathbf{r}) \hat{a}_{in}$
	\item Mode space is divided into low- and high- energy subsets $L$ and $H$,
		depending on whether $\hbar^2 k_n^2 / 2 m $ is more or less than $\epsilon_{\textrm{cut}}$~\cite{norrie}
	\item Projector operators:	
\[ P \equiv \sum\limits_{n \in L} \lvert n \rangle \langle n \rvert,\,
Q \equiv \sum\limits_{n \in H} \lvert n \rangle \langle n \rvert \]
	\item Commutation relations for restricted field operators $\hat{\psi}_{iP} \equiv P[\hat{\psi}_i]$:
\[
\left[ \hat{\psi}_{iP}(\mathbf{r}), \hat{\psi}_{jP}(\mathbf{r}^\prime) \right] =
\left[ \hat{\psi}_{iP}^\dagger(\mathbf{r}), \hat{\psi}_{jP}^\dagger(\mathbf{r}^\prime) \right] = 0
\]
\[
\left[ \hat{\psi}_{iP}(\mathbf{r}), \hat{\psi}_{iP}^\dagger(\mathbf{r}^\prime) \right] = \delta_{ij} \delta_{P}(\mathbf{r}, \mathbf{r}^\prime)
\]
	Restricted delta function:
\[
\delta_{P}(\mathbf{r}, \mathbf{r}^\prime) = \sum\limits_{n \in L} \phi_{n}^* (\mathbf{r}) \phi_{n} (\mathbf{r}^\prime).
\]
	\item If $\epsilon_{\textrm{cut}}$ is sufficiently small, contact potential can be used~\cite{morgan}:
\begin{align*}
\begin{split}
\int d\mathbf{r} \int d\mathbf{r}^\prime \,
	\hat{\psi}_i^\dagger (\mathbf{r}) \hat{\psi}_j^\dagger (\mathbf{r}^\prime)
	U_{ij}(\mathbf{r} - \mathbf{r}^\prime)
	\hat{\psi}_i(\mathbf{r}^\prime) \hat{\psi}_j(\mathbf{r}) \\
\rightarrow g_{ij} \int d\mathbf{r} \,
	\hat{\psi}_{iP}^\dagger (\mathbf{r}) \hat{\psi}_{jP}^\dagger (\mathbf{r})
	\hat{\psi}_{iP}(\mathbf{r}) \hat{\psi}_{jP}(\mathbf{r}),
\end{split}
\end{align*}
	where $g_{ij} = 4 \pi \hbar^2 a_{ij} / m$

\end{itemize}

\tiny{ \begin{thebibliography}{9}
	\bibitem{morgan} Morgan, S. A., \textit{J. Phys. B: At. Mol. Opt. Phys.} \textbf{33} 3847 (2000)
	\bibitem{norrie} Norrie, A. and Ballagh, R. and Gardiner, C., \textit{Phys. Rev. A} \textbf{73} 043617 (2006)
\end{thebibliography} }

\end{block}

\begin{block}{Master equation}

\begin{itemize}
	\item Born-Markov master equation:
	\[
		\frac{d\rho}{d t} = - \frac{i}{\hbar} [\hat{H}, \rho] +
		\sum\limits_n \frac{\kappa_n}{2} \int\limits d\mathbf{r} L [\hat{O}_n]
	\]
% FIXME: Jack only described loss operator for three-body recombination;
% probably some global reference is necessary
	\item Loss operator for the restricted basis field~\cite{jack}:
		\[ L [\hat{O}] = 2  \hat{O} \rho \hat{O}^\dagger - \hat{O}^\dagger \hat{O} \rho - \rho \hat{O}^\dagger \hat{O} \]
	\item Operators $\hat{O}_n$ describe bath couplings with corresponding strengths $\kappa_n$
	\item There are three dominant sources of losses~\cite{burt}~\cite{mertes}:
	\begin{enumerate}
		\item Three-body recombination: $\hat{O}_1 = \hat{\psi}_{1P}^3$
		\item Two-body $\vert1\rangle$-$\vert2\rangle$ inelastic collisions: $\hat{O}_2 = \hat{\psi}_{1P} \hat{\psi}_{2P}$
		\item Two-body $\vert2\rangle$-$\vert2\rangle$ inelastic collisions: $\hat{O}_3 = \hat{\psi}_{2P}^2$
	\end{enumerate}
\end{itemize}

\tiny{ \begin{thebibliography}{9}
	\bibitem{burt} Burt, E. and Ghrist, R. and Myatt, C. and Holland, M. and Cornell, E. and Wieman, C.,
		\textit{Phys. Rev. Lett.} \textbf{79} 337--440 (1997)
	\bibitem{jack} Jack, M., \textit{Phys. Rev. Lett.} \textbf{89} 140402 (2002)
	\bibitem{mertes} Mertes, K. and Merrill, J. and Carretero-Gonz\'alez, R. and Frantzeskakis, D. and Kevrekidis, P. and Hall, D.,
		\textit{Phys. Rev. Lett.} \textbf{99} 190402 (2007)
\end{thebibliography} }

\end{block}

} % First column

% Second column
\mycolumn{
            
\begin{block}{Wigner truncation, FPE and stochastic equations}

\begin{itemize}
	\item Wigner representation~\cite{gardiner} is used for solving the master equation
	\item Derivatives of order higher than 2 can be truncated if number of modes in low-energy subset $L$
		is much less than the number of atoms in the cloud~\cite{norrie}~\cite{sinatra}
	\item Resulting Fokker-Planck equation:
\begin{equation*}
\begin{split}
\frac{\partial W}{\partial t} = & \int d\mathbf{r}\, \left\{
	- \frac{\delta}{\delta \psi_{1P}} A_1 -
	\frac{\delta}{\delta \psi_{2P}} A_2 -
	\frac{\delta}{\delta \psi^*_{1P}} A^*_1 -
	\frac{\delta}{\delta \psi^*_{2P}} A^*_2
\right. \\
& \left. + \frac{1}{2} \left(
	\frac{\delta^2}{\delta \psi_1 \delta \psi_1^*} D_{11} +
	\frac{\delta^2}{\delta \psi_2 \delta \psi_2^*} D_{22} +
	\frac{\delta^2}{\delta \psi_1 \delta \psi_2^*} D_{12} +
	\frac{\delta^2}{\delta \psi_1^* \delta \psi_2} D^*_{12} \right)
\right\} W,
\end{split}
\end{equation*}
where:
\begin{align*}
A_1 = {} & -\frac{i}{\hbar} \left( \hat{K} + g_{11} \lvert \psi_{1P} \rvert^2 + g_{12} \lvert \psi_{2P} \rvert^2 \right) \psi_{1P} \\
	& - \left( \frac{3 \kappa_1}{2} \lvert \psi_{1P} \rvert^4 + \frac{\kappa_2}{2} \lvert \psi_{2P} \rvert^2 \right) \psi_{1P} -
	\frac{i}{2} \left( \tilde{\Omega} e^{i \omega t} + \tilde{\Omega}^* e^{-i \omega t} \right) \psi_{2P}, \\
A_2 = {} & -\frac{i}{\hbar} \left( \hat{K} + V_{\textrm{hf}} +
	g_{12} \lvert \psi_{1P} \rvert^2 + g_{22} \lvert \psi_{2P} \rvert^2 \right) \psi_{2P} \\
	& - \left( \frac{\kappa_2}{2} \lvert \psi_{1P} \rvert^2 + \kappa_3 \lvert \psi_{2P} \rvert^2 \right) \psi_{2P} -
	\frac{i}{2} \left( \tilde{\Omega} e^{i \omega t} + \tilde{\Omega}^* e^{-i \omega t} \right) \psi_{1P}, \\
D_{11} = {} & 9 \kappa_1 \lvert \psi_1 \rvert^4 + \kappa_2 \lvert \psi_2 \rvert^2, \\
D_{22} = {} & \kappa_2 \lvert \psi_1 \rvert^2 + 4 \kappa_3 \lvert \psi_2 \rvert^2, \\
D_{12} = {} & \kappa_2 \psi_1 \psi_2^*
\end{align*}
	\item It is easier to solve these equations in real variables; diffusion matrix turns out to be decomposable,
		which allows one to write down stochastic equations:
% FIXME: in fact, D may not be positive-definite - the only thing we know for sure is that it is decomposable
% In addition: if we can drop out D_{12}, things get much simpler
\[ d \vec{\psi} = P \left[ \vec{A} dt + B d\vec{W} \right], \]
where
\[ \vec{\psi} = \begin{pmatrix} \textrm{Re} \psi_1 & \textrm{Im} \psi_1 & \textrm{Re} \psi_2 & \textrm{Im} \psi_1 \end{pmatrix}^T, \]
\[ \vec{A} = \begin{pmatrix} \textrm{Re} A_1 & \textrm{Im} A_1 & \textrm{Re} A_2 & \textrm{Im} A_1 \end{pmatrix}^T, \]
\[
\newcommand{\ta}{D_{11}}
\newcommand{\tb}{\textrm{Re}D_{12}}
\newcommand{\tc}{\textrm{Im}D_{12}}
\newcommand{\td}{D_{22}}
B B^T = \frac{1}{4} \begin{pmatrix}
\ta & \tb & 0 & \tc \\
\tb & \td & -\tc & 0 \\
0 & -\tc & \ta & \tb \\
\tc & 0 & \tb & \td
\end{pmatrix}
% FIXME: Matrix in explicit form is too large and does not fit in the column
%B = \frac{1}{2} \begin{pmatrix}
%\sqrt{\ta} & 0 & 0 & 0 \\
%0 & \tc \sqrt{\frac{\ta}{\ta \td - (\tb)^2}} & \sqrt{\ta + \frac{\ta (\tc)^2}{(\tb)^2 - \ta \td}} & 0 \\
%\frac{\tb}{\sqrt{\ta}} & \sqrt{\frac{\ta \td - (\tb)^2}{\ta}} & 0 & 0 \\
%-\frac{\tc}{\sqrt{\ta}} & \frac{\tb \tc}{\ta} \sqrt{\frac{\ta}{\ta \td - (\tb)^2}} &
%\frac{\tb}{\ta} \sqrt{\ta + \frac{\ta (\tc)^2}{(\tb)^2 - \ta \td}} & \sqrt{\frac{\ta \td - (\tb)^2 - (\tc)^2}{\ta}}
%\end{pmatrix}
\]
and $\vec{W}$ is the multicomponent Wiener process
\end{itemize}

\tiny{ \begin{thebibliography}{9}
	\bibitem{gardiner} Gardiner C., Quantum Noise. \textit{Springer-Verlag}, 1992
	\bibitem{norrie} Norrie, A. and Ballagh, R. and Gardiner, C., \textit{Phys. Rev. A} \textbf{73} 043617 (2006)
	\bibitem{sinatra} Sinatra, A. and Lobo, C. and Castin, Y., \textit{J. Phys. B: At. Mol. Opt. Phys.} \textbf{35} 3599-3631 (2002)
\end{thebibliography} }

\end{block}

\begin{block}{Initial state and sampling}

\begin{itemize}
	\item Initial state for simulation~\cite{scott}:
\[
\psi_{iP}(\mathbf{r}) = P[\psi_{iGP}(\mathbf{r})] + \frac{1}{\sqrt{2V}} \sum\limits_{n \in L} \chi_n e^{i \mathbf{k}_n \mathbf{r}},
\]
		where $\psi_{iP}$ is the steady solution of Gross-Pitaevskii equation,
		and $\chi_n$ are independent normally distributed complex random numbers,
		$\langle \chi_m^* \chi_n \rangle = \delta_{mn}$
	\item Second term stands for quantum fluctuations; total number of virtual particles is $M/2$,
		where $M$ is the number of low-energy modes ($L$ subspace)
	\item Moments of $\psi_{iP}$ for Wigner representation are equal to symmetrically ordered operator products
	\item Basic measurable: number of particles
\begin{align*}
N_i & = \int \langle \hat{\psi}_{iP}^\dagger \hat{\psi}_{iP} \rangle d\mathbf{r} \\
& = \int \left( \langle \left\{ \hat{\psi}_{iP}^\dagger \hat{\psi}_{iP} \right\}_{\textrm{sym}} \rangle -
	\frac{1}{2} \delta_P(\mathbf{r}, \mathbf{r})  \right) d\mathbf{r} \\
& = \int \langle \psi_{iP}^* \psi_{iP} \rangle_{\textrm{paths}} d\mathbf{r} - \frac{M}{2},
\end{align*}
	where $\langle \rangle_{\textrm{paths}}$ stands for the average over different simulation paths,
	with different random number sets for initial state and Wiener processes

\end{itemize}

\tiny{ \begin{thebibliography}{9}
	\bibitem{scott} Scott, R. and Hutchinson, D. and Gardiner, C., \textit{Phys. Rev. A} \textbf{74} 053605 (2006)
\end{thebibliography} }
\end{block}

} % Second column

\end{columns}
\end{frame}
\end{document}
